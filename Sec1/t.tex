\documentclass{beamer}
\usepackage{lmodern}
\usepackage{tikz}
\usetikzlibrary{shapes.callouts,shadows, calc}
\usepackage{listings}

\usetheme{CambridgeUS}

\tikzset{note/.style={rectangle callout, rounded corners,fill=gray!20,drop shadow,font=\footnotesize}}    

\newcommand{\tikzmark}[1]{\tikz[overlay,remember picture] \node (#1) {};}    

\newcounter{image}
\setcounter{image}{1}

\makeatletter
\newenvironment{btHighlight}[1][]
{\begingroup\tikzset{bt@Highlight@par/.style={#1}}\begin{lrbox}{\@tempboxa}}
{\end{lrbox}\bt@HL@box[bt@Highlight@par]{\@tempboxa}\endgroup}

\newcommand\btHL[1][]{%
  \begin{btHighlight}[#1]\bgroup\aftergroup\bt@HL@endenv%
}
\def\bt@HL@endenv{%
  \end{btHighlight}%   
  \egroup
}
\newcommand{\bt@HL@box}[2][]{%
  \tikz[#1]{%
    \pgfpathrectangle{\pgfpoint{0pt}{0pt}}{\pgfpoint{\wd #2}{\ht #2}}%
    \pgfusepath{use as bounding box}%
    \node[anchor=base west,rounded corners, fill=green!30,outer sep=0pt,inner xsep=0.2em, inner ysep=0.1em,  #1](a\theimage){\usebox{#2}};
  }%
   %\tikzmark{a\theimage} <= can be used, but it leads to a spacing problem
   % the best approach is to name the previous node with (a\theimage)
 \stepcounter{image}
}
\makeatother


\lstset{language=C++,
                basicstyle=\footnotesize\ttfamily,
                keywordstyle=\footnotesize\color{blue}\ttfamily,
                moredelim=**[is][\btHL]{`}{`},
}

\begin{document}
\begin{frame}[fragile]{Better approach}
	\begin{lstlisting}
#include <future>
std::map<std::string,std::string> french
{{"hello","bonjour"},{"world","tout le monde"}};
int main()
{
std::string greet=french["hello"];
auto f=std::async(`[&]{std::cout << greet <<", ";}`);
`std::string audience=french["word"];`
f.get();
std::cout<<audience<<std::endl;
}
\end{lstlisting}
	% To insert the annotation
	\begin{tikzpicture}[remember picture,overlay]
		% first annotation
		\coordinate (aa) at ($(a1)+(5,5)$); % <= adjust this parameter to move the position of the annotation 
		\node[note,draw,callout relative pointer={($(aa)-(9,3.35)$)},right] at (aa) {time consuming I/O};

		%second annotation
		\coordinate (bb) at ($(a2)+(3.25,3.25)$); % <= adjust this parameter to move the position of the annotation 
		\node[note,draw,callout relative pointer={($(bb)-(7,1.1)$)},right] at (bb) {next lookup};
	\end{tikzpicture}

\end{frame}
\end{document}
