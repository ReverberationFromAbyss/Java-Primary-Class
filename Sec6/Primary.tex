\documentclass[en, 11pt, xcolor=dvipsnames]{beamer}
\usepackage[en]{infocoBeamer}

\definecolor{hrefcol}{RGB}{0, 0, 255} % Example: blue color

% -------------------------------------------------------------------------------------
\usepackage{tikz}
\usetikzlibrary{shapes.callouts, shadows, calc}
\usepackage{listings}

%\usepackage{fontspec-xetex}

\tikzset{note/.style={rectangle collout, rounded corners,fill=grap!20, drop shadow, font=\footnotesize}}

\newcommand{\tikzmark}[1]{\tikz[overlay,remember picture] \node (#1) {};}    

\newcounter{image}
\setcounter{image}{1}

\makeatletter
\newenvironment{btHighlight}[1][]
{\begingroup\tikzset{bt@Highlight@par/.style={#1}}\begin{lrbox}{\@tempboxa}}
{\end{lrbox}\bt@HL@box[bt@Highlight@par]{\@tempboxa}\endgroup}

\newcommand\btHL[1][]{%
  \begin{btHighlight}[#1]\bgroup\aftergroup\bt@HL@endenv%
}
\def\bt@HL@endenv{%
  \end{btHighlight}%   
  \egroup
}

\newcommand{\bt@HL@box}[2][]{%
  \tikz[#1]{%
    \pgfpathrectangle{\pgfpoint{0pt}{0pt}}{\pgfpoint{\wd #2}{\ht #2}}%
    \pgfusepath{use as bounding box}%
    \node[anchor=base west,rounded corners, fill=green!30,outer sep=0pt,inner xsep=0.2em, inner ysep=0.1em,  #1](a\theimage){\usebox{#2}};
  }%
   %\tikzmark{a\theimage} <= can be used, but it leads to a spacing problem
   % the best approach is to name the previous node with (a\theimage)
 \stepcounter{image}
}
\makeatother

%\setmainfont{FiraCode Nerd Font Mono}
%\setsansfont{FiraCode Nerd Font Mono}
%\setmonofont{FiraCode Nerd Font Mono}

%\setCJKmainfont[BoldFont = STLibian-SC-Regular,]{TpldKhangXiDictTrial}
%\setCJKsansfont{DFWaWaSC-W5}
%\setCJKmonofont{STXingkai-SC-Light}
%\setCJKfamilyfont{qingsong}{FZQKBYSJW--GB1-0}
%\lstset{language=Java,
%        basicstyle=\footnotesize\ttfamily,
%        keywordstyle=\footnotesize\color{blue}\ttfamily,
%        moredelim=**[is][\bthl]{`}{`},
%}

\lstset{
  basicstyle            =   \scriptsize\ttfamily,          % 基本代码风格
    keywordstyle        =   \bfseries,          % 关键字风格
    commentstyle        =   \scriptsize\tt\itshape,  % 注释的风格,斜体
    stringstyle         =   \tt\itshape,  % 字符串风格
    flexiblecolumns,                % 别问为什么,加上这个
    numbers             =   left,   % 行号的位置在左边
    showspaces          =   false,  % 是否显示空格,显示了有点乱,所以不现实了
    numberstyle         =   \ttfamily,    % 行号的样式,小五号,tt等宽字体
    showstringspaces    =   false,
    captionpos          =   t,      % 这段代码的名字所呈现的位置,t指的是top上面
    frame               =   lrtb,   % 显示边框
}

\lstdefinestyle{Java}{
    language        =   Java,
    commentstyle    =   \scriptsize\color{red}\ttfamily,
    basicstyle      =   \scriptsize\ttfamily,
    keywordstyle    =   \scriptsize\color{blue}\bfseries,
    keywordstyle    =   [2] \color{teal},
    stringstyle     =   \scriptsize\color{magenta}\tt,
    numberstyle     =   \tiny\ttfamily,
    showstringspaces=   false,
    flexiblecolumns,                % 别问为什么,加上这个
    breaklines      =   true,   % 自动换行,建议不要写太长的行
    columns         =   fixed,  % 如果不加这一句,字间距就不固定,很丑,必须加
    basewidth       =   0.15cm,
		moredelim       =   **[is][\btHL]{`}{`},
    captionpos      =   t,      % 这段代码的名字所呈现的位置,t指的是top上面
    frame           =   lrtb,   % 显示边框
}
% -------------------------------------------------------------

% meta-data
\title{Java Programming Language \\ Simple Guide} % The short title in the optional parameter appears at the bottom of every slide, the full title in the main parameter is only on the title page
\subtitle{Infoco Programming Classes} % Presentation subtitle, remove this command if a subtitle isn't required
\author{Ug. Sihang Sha \and Pg. Muxi Qiao} % Presenter name(s), the optional parameter can contain a shortened version to appear on the bottom of every slide, while the main parameter will appear on the title slide
\institute{Xiann' Jiaotong Livepool University \\ \smallskip \textit{infoco@xjtlu.edu.cn}} % Your institution, the optional parameter can be used for the institution shorthand and will appear on the bottom of every slide after author names, while the required parameter is used on the title slide and can include your email address or additional information on separate lines
\date{\today} % Presentation date or conference/meeting name, the optional parameter can contain a shortened version to appear on the bottom of every slide, while the required parameter value is output to the title slide

%-------------------------------------------------------------------------------------------------------

% document body
% ------------------------------------------------------------------------------------------------------
\begin{document}

\maketitle

% ------------------------------------------------------------------------------------------------------

%------------------------------------------------
\section{Branch sentences -- Different condition, different response}

\subsection{Nested If}
\begin{frame}[fragile]
	\frametitle{And Nested if}

	%% Contents

	% we've introduced if - then - else sentences
	% last class
	% and we tried how to use them also
	%
	% now let us investigate it more deeper
	%
	% first thing we'd like to tell you is that, if ... else ... sentences
	% is actually designed to have only one sentence for each part
	%
	% but in last class,
	% we did write mutiple lines in then part or else part
	% how do we archive it?
	%
	% let's watch the ppt
	% do you find some difference between today's slide and last class's?
	%
	% correctly! we have barckets around <then-part> and <else-part> before
	% that result in mutiple lines to be written in the two parts
	%
	% so, what are those barckets? and how can they influence how code to be
	% evaluated?
	%
	% the barckets, which encase the code, and the code within them, are called
	% "code block"
	% they are seen as a whole sentence, thus,
	% if and else will execute altogther
	%
	% so, as we menthioned above, if and else execute one sentence separately
	% and if ... else ... is a sentence itself
	%
	% so, if and else can be nested actually
	%
	% let's look at the slide
	%
	% The example here uses nested if ... else ... sentence
	% the outer one use barckets to wrap the internal code
	% while the inner one doesn't
	%
	% we can also see that, with barckets, the code seems more
	% ordered
	%
	% So, we encourage you to use barckets even if
	% the code you'd like to execute is only one line
	%

	\begin{columns}[c]
		\begin{column}{0.98\textwidth}

			\begin{lstlisting}[style=Java]
if (<condition>)
  <then-part>
else
  <else-part>\end{lstlisting}


		\end{column}
	\end{columns}

\end{frame}

\begin{frame}[fragile]

	\begin{columns}[c]
		\begin{column}{0.98\textwidth}

			\begin{lstlisting}[style=Java]
import java.util.Scanner;
public class Sample {
  public static void main(String[] args) {
    Scanner sc = new Scanner(System.in);
    int val = sc.nextInt();
    if (val > 9) {
      if (val < 100) 
        System.out.println("The Number has two digits");
      else
        System.out.println("The Number has more digits");
    } else {
      System.out.println("The Number is too small");
    }
  } // ! main
}\end{lstlisting}

		\end{column}
	\end{columns}
\end{frame}

\subsubsection{Special case of if \dots else \dots}
\begin{frame}[fragile]
	\frametitle{Continuous if \dots else \dots}

	%% Contents:

	% We have a strategy is that put another if sentence on the else part of last if ... else ...
	% which is used to compare things separately.
	% For example,
	% the code we presented here shows how we determine wether a number is odd
	% it uses the feature we mentioned above

	\begin{columns}[c]
		\begin{column}{0.98\textwidth}

			\begin{lstlisting}[style=Java]
import java.util.Scanner;
public class Sample {
  public static void main(String[] args) {
    Scanner sc = new Scanner(System.in);
    int val = sc.nextInt();
    if (val == 0) {
      System.out.println("It's nighter even or odd");
    } else if (val % 2 == 0) {
      System.out.println("It's even");
    } else {
      System.out.println("It's odd");
    }
  } // ! main
}
\end{lstlisting}

		\end{column}
	\end{columns}

\end{frame}

\subsection{Switch case --- If \dots else \dots ? No! its to long}
\begin{frame}[fragile]
	\frametitle{Switch case --- If \dots else \dots ? No! its to long}

	%% Contents:

	% However,
	% if we need to compare many things
	% for example
	% like our homework
	% we need to compare so many things that if we use if ... else ... sentence
	% may take too much lines
	%
	% then we can take another mechanism,
	% swatch ... case ...
	%
	% in this slide
	% we can see an example
	% like this
	%
	% in general, what do you think it will be execute?
	% if we insert 1, and then it will produce 1?
	% let's try
	%
	% see?
	% we printed 1, 2, 3
	%
	% so, let's introduce switch case
	% briefly
	% first, just like if else, it has a condition
	% but what different to the formar
	% the condition switch case need is a value
	% like the c in this example
	%
	% and if it get the value
	% switch case will compare it with each case label
	% look carefully
	% we can see that every case is followed with a value
	%
	% if match, and the program will begin to execute it
	% and directly through all cases

	\begin{columns}[c]
		\begin{column}{0.98\textwidth}

			\begin{lstlisting}[style=Java]
import java.util.Scanner;
public class Sample {
  public static void main(String[] args) {
    Scanner sc = new Scanner(System.in);
    char c = sc.nextChar();
    switch (c) {
      default:
      case '1':
        System.out.println("1");
      case '2':
        System.out.println("2");
    }
  }
}
\end{lstlisting}

		\end{column}
	\end{columns}

\end{frame}

\subsubsection{Break --- Stop! Don't go forward!}
\begin{frame}[fragile]
	\frametitle{Break the case}

	%% Contents:

	% How it's going?
	% that's becouse switch will not exit automatic
	% so, that why we need break
	%
	% let's see another example
	%
	% As we mentioned above, if we execute,
	% we'd expect it to run directly across all the case
	% right?
	%
	% let's run the example
	%
	% then, we can see that
	% if we input 1
	% it produce only 1
	%
	% why?
	%
	% that's because of break
	%
	% every time we meet break, we'd like to exit this switch
	% case sentence

	\begin{columns}[c]
		\begin{column}{0.98\textwidth}

			It will work properly with break;

			\begin{lstlisting}[style=Java]
import java.util.Scanner;
public class Sample {
  public static void main(String[] args) {
    Scanner sc = new Scanner(System.in);
    char c = sc.nextChar();
    switch (c) {
      default:
      case '1':
        System.out.println("1");
        break;
      case '2':
        System.out.println("2");
        break;
    }
  }
}
\end{lstlisting}

		\end{column}
	\end{columns}
\end{frame}

\subsubsection{default --- what if every case won't be matched}
\begin{frame}[fragile]
	\frametitle{Default label}

	%% Contents:

	% I don't know wether you notied that
	% we have not only the case label
	% we have also default label
	%
	% so, what is it's usage?
	%
	% it is used for "default" branch,
	% it will be jumped to if the return value didn't match
	% any case label
	%
	% like this
	%
	% because we do not have break between default and case '1'
	% so, we have it default to print 1


	\begin{columns}[c]
		\begin{column}{0.98\textwidth}

			\begin{lstlisting}[style=Java]
import java.util.Scanner;
public class Sample {
  public static void main(String[] args) {
    Scanner sc = new Scanner(System.in);
    char c = sc.nextChar();
    switch (c) {
      `default`:
      case '1':
        System.out.println("1");
        break;
      case '2':
        System.out.println("2");
        break;
    }
  }
}
\end{lstlisting}

		\end{column}
	\end{columns}
\end{frame}

\subsection{Conditional Operator --- A simpler way to calculate Variables}
\begin{frame}[fragile]
	\frametitle{Conditional Operatior}

	%% Contents

	% Other then those mutiple-line Conditional Sentences
	% we have also conditional operator
	%
	% it is a tiny branching sentence within a line
	% like this shown in the slide
	%
	% the program shown here shows how to produce a value always active
	% so that we have no need to write if ... else ...
	%
	% the sentence show after is its structure
	%
	% please remember one thing,
	% the conditional operator returns a value and cannot be used like if ... else ...

	\begin{columns}[c]
		\begin{column}{0.98\textwidth}

			\begin{lstlisting}[style=Java]
import java.util.Scanner;
public class Sample {
  public static void main(String[] args) {
    Scanner sc = new Scanner(System.in);

    int var = sc.nextInt();
    int bar = var > 0 ? var : -var;

    System.out.println(bar);

  }
}
\end{lstlisting}
			\begin{lstlisting}[style=Java]
<condition> ? <then-part> : <else-part>;
\end{lstlisting}
		\end{column}
	\end{columns}
\end{frame}

% -----------------------------------------------
\section{Vectors --- Actually, Array}
\begin{frame}[fragile]

	%% Contents:

	% With preknowledge
	% we can know that there exist different type of variables
	% however, we still processing single Variable each time till now
	% so
	% we'd like to introduce
	% arrays
	%
	% arrays are something that stores many data, with has same type, with one entry
	%
	% for example
	% here is an array that stores int, whose name is bar, and length is 10
	%
	% so, we can learn from it that array can be mantipulated as other variables
	%
	% but, as it stores ten integer, how can we visit them?
	% we visit them by index
	% look at below
	% we can find that index starts at 0 and ends at length - 1
	%
	% by the way, the array itself is a wrapper class
	% which means, it is just like objects of Integer or str,
	% can be mantipulated by methods and proterties it provide
	%
	% so, we can get the length of a array by .length property
	%
	% but, we only mantipulate it directly for each element,
	% how can we visit every elements in an array for one time?
	%
	% That's we'll introduce next

	\begin{columns}[c]
		\begin{column}{0.98\textwidth}

			\begin{lstlisting}[style=Java]
int[] arr = new int[10];
\end{lstlisting}

			\begin{lstlisting}[style=Java]
arr[0] = 20;
arr[1] = 12;
arr[9] = 8;
\end{lstlisting}

			\begin{lstlisting}[style=Java]
System.out.println(arr.length);
\end{lstlisting}


		\end{column}
	\end{columns}

\end{frame}

% -------------------------------------------------

\section{Loop -- When we want to execute something repeatedly}
\begin{frame}[fragile]
	\frametitle{Loop --- When we want to execute something repeatly}

	%% Contents:

	% As we already know how to sperate the program's execution flow
	%
	% we need to execute some values repeatly now
	% for example, if we want to give some one a good welcome
	% and want to say hello for him a thousend times
	%
	% Shall we write like this?
	%
	% Or, consider
	% if we want to say hello to him, in different times?
	% we cannot write it all right?
	% even with switch case
	% it can always exist conditions we cannot reach

	\begin{columns}[c]
		\begin{column}{0.98\textwidth}

			\begin{lstlisting}[style=Java]
import java.util.Scanner;
public class Sample {
  public static void main(String[] args) {
    System.out.println("Hello QwQ");
    System.out.println("Hello QwQ");
    // ... 997 lines...
    System.out.println("Hello QwQ");
  }
}\end{lstlisting}


		\end{column}
	\end{columns}
\end{frame}

\subsection{While --- while condition do something}
\begin{frame}[fragile]
	\frametitle{while loop}

	%% Contents:

	% That's why we need loop
	%
	% The first loop we'd like to introduce is While loop
	%
	% just as its name
	% while something is true, and then we do something ...
	%
	% look at the slide
	%
	% this is a program that prints ten hello
	% stare at the structure,
	% what it looks like?
	% yes, it has a pair of barckets
	% which contains loop condition
	% if the condition is true
	% and the sentence follow the while will be executed,
	% repeatly
	%
	% so, we can know that,
	% just like if ... else ...
	% large barckets is not mandatory

	\begin{columns}[c]
		\begin{column}{0.9\textwidth}

			\begin{lstlisting}[style=Java]
import java.util.Scanner;
public class Sample {
  public static void main(String[] args) {
    int val = 10;
    while (val > 0) {
      System.out.println("Hello");
      val --;
    }
  }
}\end{lstlisting}

		\end{column}
	\end{columns}

\end{frame}

\subsubsection{Loop --- do something and check wether to continue}
\begin{frame}[fragile]
	\frametitle{Do \dots While \& Loop Rounds}
	\framesubtitle{Important to remember that code are executed linearly}

	%% Contents:

	% we can check condition before we execute the code
	% we can also execute the code first and then check wether need to continue
	%
	% think about it, what if we want to print 1 to 10?
	% we can use while indeed
	% but how about do while?
	%
	% let's look at the slide
	% we put a word 'do' here first
	% and then is our code to be executed
	% wrapped by large barckets
	% last, there is our while sentence and condition
	% don't forget to put a semi-colon at the end of while sentene
	%

	\begin{columns}[c]
		\begin{column}{0.9\textwidth}

			\begin{lstlisting}[style=Java]
import java.util.Scanner;
public class Sample {
  public static void main(String[] args) {
    int val = 1;
    do{
      System.out.println(val);
    } while (val <= 10);
  }
}\end{lstlisting}

		\end{column}
	\end{columns}

\end{frame}

\subsubsection{Break --- Jump out of the loop}
\begin{frame}[fragile]
	\frametitle{Jump, out of the loop!}

	%% Contents

	% Just like we are mantipulate switch case sentences
	% we can also use break in the loops
	% and, what is more,
	% break in loop can jump out of nested loops
	%
	% first, we can give each loop a lable
	% just like case lable in switch case
	% and then, use break label; to jump out of the loop
	% 
	% look at the slide

	\begin{columns}[c]
		\begin{column}{0.9\textwidth}

			\begin{lstlisting}[style=Java]
import java.util.Scanner;
public class Sample {
  public static void main(String[] args) {
    int a = 0;
    while (a < 100) {
      while (a < 50) {
        System.out.println(a);
        a ++;
        break;
      }
      System.out.println(a);
      a ++;
    }
  }
}\end{lstlisting}

		\end{column}
	\end{columns}

\end{frame}

\subsubsection{Break --- Jump out of the loop}
\begin{frame}[fragile]
	\frametitle{Jump, out of the loop!}

	%% Contents

	% you can tell the difference between two example
	%
	% and we can even use break within switch to jump out of the loop

	\begin{columns}[c]
		\begin{column}{0.9\textwidth}

			\begin{lstlisting}[style=Java]
import java.util.Scanner;
public class Sample {
  public static void main(String[] args) {
    int a = 0;
    loop:
    while (a < 100) {
      while (a < 50) {
        System.out.println(a);
        a ++;
        break loop;
      }
      System.out.println(a);
      a ++;
    }
  }
}\end{lstlisting}

		\end{column}
	\end{columns}

\end{frame}

\subsubsection{Continue --- Next circle}
\begin{frame}[fragile]
	\frametitle{Next circle, next round, but the same loop}

	%% Contents:

	% except break,
	% we have continue as well
	%
	% what is the difference?
	% the main difference they have is that,
	% break will jump out of loop, and even if the condition is still true
	% but continue just ignore rest code after continue, 
	% and execute other loop if condition is true;
	%
	% continue always work for innerest loop

	\begin{columns}[c]
		\begin{column}{0.9\textwidth}

			\begin{lstlisting}[style=Java]
import java.util.Scanner;
public class Sample {
  public static void main(String[] args) {
    int a = 0;
    while (a < 100) {
      while (a < 50) {
        System.out.println(a);
        a ++;
        continue;
      }
      System.out.println(a);
      a ++;
    }
  }
}\end{lstlisting}

		\end{column}
	\end{columns}

\end{frame}

\subsection{For --- Another loop}
\begin{frame}[fragile]
	\frametitle{For something, and do something}
	\framesubtitle{Another way to visit all elements in an array}

	%% Contents:

	% as we introduced while loop
	% there exist another type of loop
	%
	% for loop
	% we may don't know what for means here
	% but just remember it is enough
	%
	% the basic structure is shown above
	%
	% first part is that "int i = 0;", along with the semi-colon
	% called initialization part, defines loop flag variables
	% on the center, there exist condition part, if the condition is true,
	% we do execute the codes
	% then is code part
	% same as before, we only execute one sentence here, and if we want to execute more
	% large barckets are needed
	% we also, encourage to write barckets every time you meet them
	% last, we comes to the update part, which shown here is "i ++"
	% every time when code are executed, then the loop will execute sentences here
	% no matter wether break or continue is called.
	% thus we usually use this part to update our loop flag
	%


	\begin{columns}[c]
		\begin{column}{0.9\textwidth}

			\begin{lstlisting}[style=Java]
import java.util.Scanner;
public class Sample {
  public static void main(String[] args) {
    for (int i = 0; i < 10; i ++) {
      System.out.println("hello world");
    }
  }
}\end{lstlisting}


		\end{column}
	\end{columns}

\end{frame}

\subsection{Foreach expression --- For something in something}
\begin{frame}[fragile]
	\frametitle{For something in something, and thus we do something}

	%% Contents:

	% for loop is usually used to visit array,
	% but for somebody, the original for loop is still too complicate
	% thus, we have foreach expression here
	%
	% instead of three part separate by semi-colon
	% we have a variable defination here
	% then followed by colon and the array
	%
	% we can try the expression
	%
	% but remember, we cannot change the
	% Contents of array but assign new value to the variable
	% which means, the variable here just clone the value of elements
	% rether then reference them

	\begin{columns}[c]
		\begin{column}{0.9\textwidth}

			\begin{lstlisting}[style=Java]
import java.util.Scanner;
public class Sample {
  public static void main(String[] args) {
    int [] arr = new int[] {9,8,7,6,5,4,3,2,1};
    for(int val : arr) {
      System.out.println(val);
    }
  }
}\end{lstlisting}

		\end{column}
	\end{columns}

\end{frame}

\section{Variable scope}
\begin{frame}[fragile]
	\frametitle{Variable scope}

	%% Contents:

	% but have you noticed?
	% if we try to visit variables defined within loop
	% or other variables defined within blocks
	%
	% the compiler will raise an error?
	%
	% that's because we have variable scope
	%
	% the inner reference to a variable will result in the program
	% try to find the variable defined in the block, if it cannot find
	% it will try to find outer
	% if it cannot find even the outest barckets, it will riase an error
	%
	% for example
	% this will indeed raise an error


	\begin{columns}[c]
		\begin{column}{0.98\textwidth}

			\begin{lstlisting}[style=Java]
import java.util.Scanner;
public class Sample {
  public static void main(String[] args) {
    for (int i = 1; i < 10; i ++) {
      System.out.println(i);
    }
    System.out.println(i); // we cannot visit it now!!
  }
}\end{lstlisting}

		\end{column}
	\end{columns}
\end{frame}

% ----------------------------------------------------------------------------------------

\section{Functions \& Methods}
\subsection{Functions}
\begin{frame}[fragile]
	\frametitle{Functions, actually is a kind of Abstraction}

	%% Contents:
	% 
	% Just as what we said before
	% the mathmathic meaning function has different expression in Java
	%
	% So, why don't we just written directly in the Program ?
	%
	% Here is a sample program that defined and used a funciton
	%
	% we can see that,
	% the function foo, do output hello and something others
	%
	% but when we call it, it just run, and we do not need to know how it is implemented
	% that we called balck box abstraction.
	%
	% Functions, apply arguments, evaluate, produce values, and finally return the result
	% all we need know is what arguments it need,
	% and what output it generate
	% we do not need to know, how it mantipulate the arguments, wether it call any other functions
	% like a black box,
	% (draw in white board)
	%
	% Actually, just as what we already know in the math
	% functions in Java, accept some parameters, and then process then,
	% return a value, however, sometimes, it can also do something like example
	% here, the function just print something onto the screen
	% and do not return a value
	%
	% this is another difference between Java and math

	\begin{columns}[c]
		\begin{column}{0.98\textwidth}

			\begin{lstlisting}[style=Java]
public class Sample {
  public static void main(String[] args) {
    foo("World");
  }

  private static void foo(String name) {
    System.out.println("hello, " + name);
  }
}
\end{lstlisting}

		\end{column}
	\end{columns}

\end{frame}

\subsubsection{Access modifier --- A way to control accessblity}
\begin{frame}[fragile]
	\frametitle{Access modifier --- A way to control accessblity}

	%% Contents:
	%
	% So we need to let us kown what,
	% Java is a oop language, which, oop means object obtain Programming
	% so, what is its meaning?
	%
	% That means, everything we mantipulated in java is object
	% every data is orgnaized by class
	% and every data is presented by object
	%
	% objects, in Java, is just like thing in our real life
	% while class, describe the common attrubtion they have
	% like World, Phylum Class Order Family Genus Species in Biology
	% each entry in Species define a common attrubtion of a creature
	%
	% so, as creatures have their private organ,
	% objects in Java also have those accessblity difference
	% so we need access-modifier to tell the program what
	% can be seen by others
	%
	% however, as we haven't fully introduce class yet
	% we won't talk about this specially
	% 
	% when we need to write a program now
	% use public is OK
	%
	%

	\begin{columns}[c]
		\begin{column}{0.9\textwidth}

			\begin{table}
				\begin{center}
					\begin{tabular}[c]{l l l}
						Access Modifier & keyword   & Meaning                           \\ \\
						public          & public    & can be visited everywhere         \\
						private         & private   & can be visited only by same class \\
						protected       & protected & can be visited only by objects    \\
						                &           & in the same package, or subclass  \\
						default         &           & only objects in the same package  \\
						                &           & can visit                         \\
					\end{tabular}
				\end{center}
			\end{table}

			\begin{lstlisting}[style=Java]
public static int foo(String name) {
  System.out.println("hello, " + name);
}
      \end{lstlisting}
		\end{column}
	\end{columns}

\end{frame}

\subsubsection{Static Methods --- Functions can be called with Class Name directly}
\begin{frame}[fragile]
	\frametitle{Static Methods --- Functions can be called with Class Name directly}

	%% Contents:

	% Static means static
	%
	% so, what is "static" function?
	%
	% put it simply
	% is some functions that can be called directly.
	% Without create an object
	%
	% compare to objects we meet before
	% for example, String str = "";
	% we mentioned that it is a object right?
	% and we can call some methods with the object
	% those methods, in contrust, are "dynamic" methods
	%
	% We cannot use those methods directly in our program
	%
	% But we also met some other functions
	% like, Integer.parseInt()
	%
	% As we menthioned before, Integer is a class, wrapper class
	% class is used to define a object, but why can we use it in function call?
	% that's is one example of our static function
	%
	%
	% So, as we mention the term class for such many times
	% what class contains?
	%
	% At last class, we say that class is a way to orgnaize data
	% but how do it orgnaize them?
	% in Java, a class has two part
	% one is proerties, they are variables, store all information
	% we think is associated with the target we want to describe
	% another is methods, they are something we are talking now
	% methods is what the target can do
	% just like cats can meow, dogs can bark, cars can be drive,
	% people can think
	% 
	% But only have orgnaize method of data is not enough
	% we need something further
	% something actually store the data
	%
	% that is why we need object,
	% objects are instance of the specified data
	%
	% Then, we can talk about the difference between function and method now
	% function, is something accept parameters to produce a result
	% method, is what a specified object can do,
	% produce a result based on both parameter and the object itself

	\begin{columns}[c]
		\begin{column}{0.9\textwidth}

			\begin{lstlisting}[style=Java]
public class Sample {
  public static void main(String[] args) {
    String str = "str";
    System.out.println(str.concat("This is a method call"));
    // Method is kind of function, which is associated with an object
    Integer.parseInt("1234");
    // While here is static function call
    foo("World");
    // We define a function, and we call it
  }
  // Our "static" function
  private `static` void foo(String name) {
    System.out.println("hello, " + name);
  }
}
\end{lstlisting}

		\end{column}
	\end{columns}

\end{frame}

\subsubsection{Define a static function}
\begin{frame}[fragile]
	\frametitle{Define a static function}

	%% Contents:

	% Here is how we declare a function
	% the upper one is the from
	% to complex?
	% don't worry
	%
	% like how we describe the variables
	% we provide example as well
	%
	% The function down here is a simplest one
	% which has only one sentence
	%
	% we can see that
	% access-modifier here is private
	% static means it is a static function
	% return-type is void

	\begin{columns}[c]
		\begin{column}{0.9\textwidth}

			\begin{lstlisting}[style=Java]
<access-modifier> [static] <return-type> <function-name>(<parameter-list>*) {
    [<sentences>;]*
    [return <return-value>;]*
}
\end{lstlisting}

			\begin{lstlisting}[style=Java]
private static void foo(String name) {
  System.out.println("hello, " + name);
}
\end{lstlisting}

		\end{column}
	\end{columns}

\end{frame}

\subsubsection{return --- Exit the function}
\begin{frame}[fragile]
	\frametitle{Return --- Exit the function}
	\framesubtitle{And return a value}

	%% Contents:

	% Here we have another example
	% add
	% the function here, compared with last one
	% have something different
	%
	% we have a return sentence
	% which means, we can return a value here
	%
	% return do two things
	% first, return a value
	% second, end the function
	%
	% every time the program meet the return sentence
	% it jump out the function
	%
	% Also, the return sentence can be not only one
	% but we encourage you to write only one return
	% in your program

	\begin{columns}[c]
		\begin{column}{0.9\textwidth}

			\begin{lstlisting}[style=Java]
private static int add(int a, int b) {
    return a + b;
}
\end{lstlisting}

		\end{column}
	\end{columns}

\end{frame}

\subsubsection{Main Function --- Where the Program Starts}
\begin{frame}[fragile]
	\frametitle{Main Function --- Where the Program Starts}
	\framesubtitle{A special convention}

	%% Contents:

	% After knowing what function is and defining your own functions
	% let's look back to the "hello, world"
	%
	% look at the slides
	% we can see that "public static void main(String[] args)" have been Highlighted
	% right?
	%
	% what is the difference between your functions and it?
	%
	% actually, none, right?
	%
	% Yes, the main is indeed a function,
	% every java function starts here
	% ,if you don't pass any arguments to the compiler
	%
	% let us investigate the main function
	% firstly, it is a public method of class Sample, and it is defined as "static",
	% which means it can be called by anyone with Sample.main
	%
	% and it returns void, is that it will not return anything
	% last, the arguments that will be passed to it are a list of String
	% We don't known what it actually is right?
	% As every function call that will be written in our program must have corresponding arguments
	% don't worry, we'll introduce it next class

	\begin{columns}[c]
		\begin{column}{0.9\textwidth}

			\begin{lstlisting}[style=Java]
public class Sample {
  `public static void main(String[] args)` {
    foo("World");
  }

  private static int foo(String name) {
    System.out.println("hello, " + name);
  }
}
\end{lstlisting}

		\end{column}
	\end{columns}

\end{frame}

% ---------------------------------------------------

\section{Stack and Heap}
\begin{frame}[fragile]
	\frametitle{Stack and Heap}

	\begin{columns}[c]
		\begin{column}{0.9\textwidth}
			something

		\end{column}
	\end{columns}
\end{frame}
% ----------------------------------------------------------------------


\QApage

% ----------------------------------------------------------------------

\section{Referencing}

\begin{frame}
	\frametitle{Citing References}

	\bigskip % Vertical whitespace

\end{frame}

% --------------------


% ----------------------------------------------------------------------

\begin{frame}
	\frametitle{Acknowledgements}

	\begin{columns}[t] % The "c" option specifies centered vertical alignment while the "t" option is used for top vertical alignment
		\begin{column}{0.45\textwidth} % Left column width
		\end{column}
		\begin{column}{0.5\textwidth} % Right column width
		\end{column}
	\end{columns}
\end{frame}

\end{document}
