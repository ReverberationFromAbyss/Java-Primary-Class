%----------------------------------------------------------------------------------------
%  PACKAGES AND OTHER DOCUMENT CONFIGURATIONS
%----------------------------------------------------------------------------------------

\documentclass[
  11pt, % Set the default font size, options include: 8pt, 9pt, 10pt, 11pt, 12pt, 14pt, 17pt, 20pt
  %t, % Uncomment to vertically align all slide content to the top of the slide, rather than the default centered
  %aspectratio=169, % Uncomment to set the aspect ratio to a 16:9 ratio which matches the aspect ratio of 1080p and 4K screens and projectors
  xcolor=dvipsnames
]{beamer}

\graphicspath{{Images/}{./}} % Specifies where to look for included images (trailing slash required)

\usepackage{booktabs} % Allows the use of \toprule, \midrule and \bottomrule for better rules in tables

\usepackage{xeCJK}
%\setCJKsansfont{黑体}
\usepackage{multicol}
\usepackage{wrapfig}

\usepackage{ctex}
\usepackage{soul}
\usepackage{xcolor}
\usepackage{color}

\usepackage{courier}
\usepackage{lmodern}
\usepackage{tikz}
\usetikzlibrary{shapes.callouts, shadows, calc}
\usepackage{listings}

\usepackage{fontspec-xetex}

\tikzset{note/.style={rectangle collout, rounded corners,fill=grap!20, drop shadow, font=\footnotesize}}

\newcommand{\tikzmark}[1]{\tikz[overlay,remember picture] \node (#1) {};}    

\newcounter{image}
\setcounter{image}{1}

\makeatletter
\newenvironment{btHighlight}[1][]
{\begingroup\tikzset{bt@Highlight@par/.style={#1}}\begin{lrbox}{\@tempboxa}}
{\end{lrbox}\bt@HL@box[bt@Highlight@par]{\@tempboxa}\endgroup}

\newcommand\btHL[1][]{%
  \begin{btHighlight}[#1]\bgroup\aftergroup\bt@HL@endenv%
}
\def\bt@HL@endenv{%
  \end{btHighlight}%   
  \egroup
}

\newcommand{\bt@HL@box}[2][]{%
  \tikz[#1]{%
    \pgfpathrectangle{\pgfpoint{0pt}{0pt}}{\pgfpoint{\wd #2}{\ht #2}}%
    \pgfusepath{use as bounding box}%
    \node[anchor=base west,rounded corners, fill=green!30,outer sep=0pt,inner xsep=0.2em, inner ysep=0.1em,  #1](a\theimage){\usebox{#2}};
  }%
   %\tikzmark{a\theimage} <= can be used, but it leads to a spacing problem
   % the best approach is to name the previous node with (a\theimage)
 \stepcounter{image}
}
\makeatother

%\setmainfont{FiraCode Nerd Font Mono}
%\setsansfont{FiraCode Nerd Font Mono}
%\setmonofont{FiraCode Nerd Font Mono}

%\setCJKmainfont[BoldFont = STLibian-SC-Regular,]{TpldKhangXiDictTrial}
%\setCJKsansfont{DFWaWaSC-W5}
%\setCJKmonofont{STXingkai-SC-Light}
%\setCJKfamilyfont{qingsong}{FZQKBYSJW--GB1-0}
%\lstset{language=Java,
%        basicstyle=\footnotesize\ttfamily,
%        keywordstyle=\footnotesize\color{blue}\ttfamily,
%        moredelim=**[is][\bthl]{`}{`},
%}

\lstset{
    basicstyle          =   \ttfamily,          % 基本代码风格
    keywordstyle        =   \bfseries,          % 关键字风格
    commentstyle        =   \tt\itshape,  % 注释的风格,斜体
    stringstyle         =   \tt\itshape,  % 字符串风格
    flexiblecolumns,                % 别问为什么,加上这个
    numbers             =   left,   % 行号的位置在左边
    showspaces          =   false,  % 是否显示空格,显示了有点乱,所以不现实了
    numberstyle         =   \zihao{-5}\ttfamily,    % 行号的样式,小五号,tt等宽字体
    showstringspaces    =   false,
    captionpos          =   t,      % 这段代码的名字所呈现的位置,t指的是top上面
    frame               =   lrtb,   % 显示边框
}

\lstdefinestyle{Java}{
    language        =   Java,
    commentstyle    =   \color{red}\ttfamily,
    basicstyle      =   \zihao{-5}\ttfamily,
    keywordstyle    =   \color{blue}\bfseries,
    keywordstyle    =   [2] \color{teal},
    stringstyle     =   \color{magenta}\tt,
    numberstyle     =   \zihao{-6}\ttfamily,
    showstringspaces    =   false,
    flexiblecolumns,                % 别问为什么,加上这个
    breaklines      =   true,   % 自动换行,建议不要写太长的行
    columns         =   fixed,  % 如果不加这一句,字间距就不固定,很丑,必须加
    basewidth       =   0.15cm,
		moredelim       =   **[is][\btHL]{`}{`},
    captionpos          =   t,      % 这段代码的名字所呈现的位置,t指的是top上面
    frame               =   lrtb,   % 显示边框
}

\usetheme{Madrid}
\usefonttheme{default} % Typeset using the default sans serif font
\usepackage{palatino} % Use the Palatino font for serif text
\usepackage[default]{opensans} % Use the Open Sans font for sans serif text

\useinnertheme{circles}

%----------------------------------------------------------------------------------------
%  PRESENTATION INFORMATION
%----------------------------------------------------------------------------------------

\title[Java Sec.2]{Java Programming Language Simple Guide \\ Java 基础课} % The short title in the optional parameter appears at the bottom of every slide, the full title in the main parameter is only on the title page

\subtitle{Infoco Computer Club Programming Classes \\ } % Presentation subtitle, remove this command if a subtitle isn't required

\author[SSH. \and MXQ. (TA)]{Ug. Sihang Sha \and Pg. Muxi Qiao} % Presenter name(s), the optional parameter can contain a shortened version to appear on the bottom of every slide, while the main parameter will appear on the title slide

\institute[XJTLU InfoCo Club]{Xiann' Jiaotong Livepool University \\ \smallskip \textit{infoco@xjtlu.edu.cn}} % Your institution, the optional parameter can be used for the institution shorthand and will appear on the bottom of every slide after author names, while the required parameter is used on the title slide and can include your email address or additional information on separate lines

\date[\today]{\today} % Presentation date or conference/meeting name, the optional parameter can contain a shortened version to appear on the bottom of every slide, while the required parameter value is output to the title slide

%----------------------------------------------------------------------------------------

\begin{document}

%----------------------------------------------------------------------------------------
%  TITLE SLIDE
%----------------------------------------------------------------------------------------

\begin{frame}
	\titlepage % Output the title slide, automatically created using the text entered in the PRESENTATION INFORMATION block above

\end{frame}


%----------------------------------------------------------------------------------------
%  TABLE OF CONTENTS SLIDE
%----------------------------------------------------------------------------------------

% The table of contents outputs the sections and subsections that appear in your presentation, specified with the standard \section and \subsection commands. You may either display all sections and subsections on one slide with \tableofcontents, or display each section at a time on subsequent slides with \tableofcontents[pausesections]. The latter is useful if you want to step through each section and mention what you will discuss.

\begin{frame}[allowframebreaks]
	\frametitle{Presentation Overview} % Slide title, remove this command for no title

	\tableofcontents % Output the table of contents (all sections on one slide)
	%\tableofcontents[pausesections] % Output the table of contents (break sections up across separate slides)

\end{frame}

%----------------------------------------------------------------------------------------
%  PRESENTATION BODY SLIDES
%----------------------------------------------------------------------------------------

\section{Variables, And Type of Variables}
%------------------------------------------------
\begin{frame}[fragile]
	\frametitle{Variables}

	%% Contents:
	%
	% 变量是我们在Java编程中最常遇到的东西. 
	%
	% 大家想一下, 计算机之所以被称作计算机,
	% 核心就在于计算两字, 而Java作为一门计算机编程语言,
	% 则提供了数以百计的对数据进行操作, 运算的方法
	%
	% 这些东西当然可以对我们写在纸面上的量,
	% 字面量(或立即值, in assembly), 进行计算, 比如 1+2
	% 但是, 就像数学的函数一样,
	% 如果只提供一个固定量, 是没有意义的
	% 这时候, 就必须要有变量来代替固定值了
	%
	% 本次的主要内容是详细讲解于变量相关的内容

	\begin{columns}[c]
		\begin{column}{0.9\textwidth}

			Variables are something we manipulate in Java often.\\

			As we mentioned last class, variable can store values, do algebra operations and draw value stored inside them for use then.

			\begin{columns}[c]
				\begin{column}{0.4\textwidth}

					literal (or immediate, in assembly)

					\begin{lstlisting}
2147483647
0x7fffffff
017777777777
1.0
2e4
\end{lstlisting}

				\end{column}

				\begin{column}{0.4\textwidth}

					variable(Definition, in Java)

					\begin{lstlisting}
int foo = 2147483647;
int bar = -2147483648;
double real = 1.0;
double var = 3e1;
\end{lstlisting}

				\end{column}
			\end{columns}

		\end{column}
	\end{columns}

\end{frame}

%------------------------------------------------
\subsection{Data, and Variables which are assigned data}

\begin{frame}[fragile]
	\frametitle{Data, and Variables which are assigned data}

	%% Contents:
	%
	% 变量是一个容器, 而它需要承装的则是 "数据"
	% 如我们刚才展示的各种整数实数,
	% 乃至于之前所介绍过的字符串
	% 这样的一些存储了信息的量, 就都可以是我们口中的 "数据"
	%
	% 我们再次以之前的 "Hello, World!" 程序为例,
	% 这个程序中, 哪些东西是数据呢?
	% 答案是 那个 "Hello, World!" 的字符串,
	% 它承载的就是这样你好世界的一个信息
	%
	% 不过大家请注意, 这个字符串, 实际上本身不是数据
	% 而是我们在前面提到了的 "字面量",
	% 那么这些我们可以直接在程序中写下的字面量,
	% 又是什么东西呢?
	% 它实际上是数据的一种表现形式, 叫值
	% 而数据是一种抽象的概念, 需要一些其他的方式来代表, 操作
	% 所以有了值这种东西, 来代表某些数据.
	% 因此, 从这个方面来看, 单独的值本身没有意义
	%
	% 而在后面, 我们也将接触除了这些可以被直接写出的 "值" 以外的对数据的表示方法
	% 
	% 不过直接写出值并无法对其进行更进一步的操作,
	% 因此也就有了变量
	% 变量, 之前已经讲过, 是一种容器,
	% 大家从直观上可以看出, 编程的变量和数学上的变量有着千丝万缕的联系
	% 比如大家可以看一下PPt上的这个示例
	%
	% 这里分别是数学中的函数和Java中的函数
	% 它们干的事情是相似的,
	% 用一个参数调用它, 并且产出一个值
	% 就像函数的定义, 对于集合X中的值, 在集合Y上的映射
	% x是函数的自变量, 产出则是因变量
	%
	% 不过, 在程序设计中, 变量的作用更进一步, 它可以承载一个值,
	% 也可以被重新赋值为其他的值,
	% 它可以作为函数的自变量
	% 也可以在函数求值的过程中被定义,
	% 承载其他表达式(可以与数学中的表达式, 或函数同等理解)的结果
	%

	\begin{columns}[c]
		\begin{column}{0.9\textwidth}

			Data, is something we manipulate in programming and contains some information.\\

			Value, is one represent for data.\\

			Variables, are some containers that hold those data.
			Or, officially, a variable is a named unit of data that is assigned a value.

			\begin{columns}[c]
				\begin{column}{0.5\textwidth}
					\[
						f(x) = x^2 + x - 8
					\]
				\end{column}
				\begin{column}{0.5\textwidth}
					\begin{lstlisting}[style=Java]
int f(int x){
  return x * x + x - 8;
}
          \end{lstlisting}

					\begin{lstlisting}[style=Java]
int d = f(8);
// which is same as 
int g = 8 * 8 + 8 - 8;
// or, even
int a = 8;
int y = a * a + a - 8;
          \end{lstlisting}
				\end{column}
			\end{columns}

		\end{column}
	\end{columns}

\end{frame}

\subsection{Variable, Definition of variables}
\begin{frame}[fragile]
	\frametitle{Definition for variables}

	%% Contents:
	% 
	% 大家早在上节课, 其实就以接触到变量的定义了
	% 不过, 在这次课中, 我们将详细讲解变量定义的方式,
	% 以及其相关的含义.
	%
	% 不过呢, 当然不包括命名规范啦
	% 毕竟在上一节课就已经讲的很清楚了
	%
	% 大家可以看到ppt上展示的第一个部分的东西
	% 展示了变量定义的规则
	% 不过可能同学们不太看得懂这些符号表示的意思,
	% 所以大家可以看下面,
	%
	% 首先, 第一行, 我们干的事情是定义了一个变量foo,
	% 这个变量的类型是int, 这里的int代表, 变量类型.
	% 后续我门将会详细讲解变量类型和变量的关系
	%
	% 注意哦, 这里和我们之前写的都不一样
	% 只是定义了变量, 而没有给它赋值
	% 当我们需要使用这个变量的时候,
	% 如果直接使用而未给它赋值, 就会出现问题
	% 只会得到它的默认值
	% 因为变量类型是int, 而int的默认值为0,
	% 所以, 如果后续不对它重新赋值,
	% 直接使用的话, 我们会得到结果0
	% 不同的变量类型会又不同的初值,
	% 不过这些将会在后面详细说明
	%
	% 下面一行, 类似的方式, 我们定义了一个int类型的变量,
	% bar, 但是与第一行不同的是, 我们在定义变量的同时,
	% 给了它一个初始的值, 这里, 这个值是10
	% 像这样对变量进行定义, 并赋初值的行为, 叫做变量的初始化.
	% 在定义变量的时候初始化是一个很好的习惯
	% 哪怕就是希望变量被初始化为0, 或者它们的默认值
	%
	% 变量也可以连续定义, 并且连续初始化
	% 就像下面两行中, 对于double类型的变量的定义

	\begin{columns}[c]
		\begin{column}{0.9\textwidth}

			\begin{lstlisting}
<Type-Identifier> <Variable-Name> [= <Initial-Value>];
// OR
<Type-Identifier> <Variable-Name> [= <Initial-Value>]
                [, <Variable-Name> [= <Initial-Value>]]
                [,...]
                ...;
// <...> means compent and its meaning
// [...] means following parts are optional
      \end{lstlisting}

			\begin{lstlisting}
int foo;      // define a variable called foo
int bar = 10; // define a variable, whose initial value is 10 and called bar
// Above we define each variable in a single line
double a,b;   // define two variable, one is a, and the other is b
double read = 1.0, pi = 3.14;
double alpha = 1.0, beta = 3.6, theta = 9.9;
      \end{lstlisting}

		\end{column}
	\end{columns}

\end{frame}

% --------------------------------------------------------

\subsection{Type of variables, and values}
\begin{frame}[fragile]
	\frametitle{Variable Type}
	% TODO: And brief introduce to class, object and instance
	% (类, 对象 和实例)
	%
	%% Contents:
	% 大家或许会注意到,
	% 我们在上一页的ppt中提到了int型, double型等字词
	% 这些其实就是变量的类型,
	%
	% 在现实生活中, 数据实际上是有类型的
	% 比如说, 这里有50个同学, 这个数据一定是个整数对吧,
	% 不会出现50.5个同学这种表述
	% 不然, 咱们这就不是编程课, 是犯罪现场了
	%
	% 而当我们去银行取钱, 则又要考虑到精度在2位小数的实数了
	% 因为除了$1 或 $20,这种, 也会有5cent或者更小的金额
	% 这时候, 如果还用整数表示, 就势必会造成误差
	% 这在涉及到交易的环境下是完全无法被接受的
	%
	% 同时, 数据不止需要有数字,
	% 有时候也需要用字符表示,
	% 比如我们的hello world, 这是没有办法用以实数, 乃至虚数来表达的
	% 这些就又需要字符和字符串类型
	%
	% 在计算机中, 有关于数据的差异更大, 相对应的也有各种不同的限制
	% 如 int只能表示从-2147483648~2147483647之间的数
	% 而 Byte甚至只能表示-1024~1023之间的数
	%
	% 下表展示了我们将学习的Java中的一些基本数据类型
	% 首先是老熟人String, 如果只说英文名不太习惯的话,
	% 大家对字符串这个名称是否有印象呢
	%
	% 然后下面的四个是我们的整数类型,
	% 包含了byte, short, int, long四种, 分别对应了四种不同的范围
	% 再然后是实型数据, 或者叫做浮点型
	% 包含float和double, 它们的特点之一是数据不一定精确, 但是却可以表示极大的实数
	%
	% 再往下, 是字符型数据, 这个东西实际上组成了字符串
	% 不过我们几乎不会与它打太深的交道,
	%
	% 最后是逻辑值, 布尔型数据, 它只有一个成员, boolean
	% 不好读是么
	% 没关系, 记住就行了,
	% 在程序设计中, 我们不会太常的与它打直接交道
	% 但是我们却会在各种涉及到做与不做的判断的地方遇见它们
	%
	\begin{columns}[c]
		\begin{column}{0.9\textwidth}

			\begin{table}
				\begin{tabular}{l l l}
					\toprule
					Type      & Defination   & default Form         \\
					\midrule
					String    & String foo;  & ""                   \\
					          &              &                      \\
					Byte      & byte foo;    & 0x0 / 0 / 00 / 0b0   \\
					Short     & short foo;   & 0x0 / 0 / 00 / 0b0   \\
					Integer   & int foo;     & 0x0 / 0 / 00 / 0b0   \\
					Long      & long foo;    & 0x0l / 0l / 00/ 0b0l \\
					          &              &                      \\
					Double    & double foo;  & 0.0d                 \\
					Float     & float foo;   & 0.0f                 \\
					          &              &                      \\
					Character & char foo;    & '$\backslash0$'      \\
					          &              &                      \\
					Boolean   & boolean foo; & false                \\

					\bottomrule
				\end{tabular}
			\end{table}


		\end{column}
	\end{columns}

\end{frame}


% --------------------------------------------------------

\subsubsection{String, More Specified}
\begin{frame}[fragile]
	\frametitle{Methods for manipulate string}

	%% Contents:
	%
	% 作为我们第一个程序里面就涉及到的数据,
	% 字符串, 不仅被用于演示程序的执行, 在程序设计过程当中也相当有用
	%
	% 所以, 针对字符串, Java实际上提供了很多相关方法
	% 比如上一节课中讲到的字符串拼接
	%
	% 对字符串的拼接有两种方式
	% 一种是我们在上一节课就已经了解
	% 了的+号, 大家可以看到ppt的第二行
	%
	% 但是, 在拼接字符串的时候, 我们还有另一种方式
	% 就是concatnate方法
	% 大家可以看到ppt上绿色高亮的部分
	% 这样一个跟在对象后面, 以点开头的东西, 就是方法的调用
	%
	% 大家或许会注意到, 这里提到了 "对象", 和 "方法", 两个概念
	% 这里是两个涉及到了Java深层设计的东西,
	% 我们将在后面的课程中详细讲解
	%
	% 正如之前所说的,
	% 如1234这样的直观上的数值是数据的一种表现形式,
	% 对象则是数据的另一种表现形式,
	% 它将与某些事物有关的东西全都放在一起,
	% 形成一个被称作 "类" 的组织关系
	% 类可以被认为是对于一些事物共有属性的概括
	% 比如说, 我们可以用"猫" 来统称所有的缅因猫, 暹罗猫等各个品种的猫
	% 用 "狗" 来统称所有的会汪汪叫, 有鼻子有眼四足行走的生物
	% 但是, 由于类其实是将我们之前提到的这些整数或者其他的数据,
	% 我们称之为 "基本数据" 用一些方式组织起来,
	% 所以, 几乎所有的用于基本数据的操作,
	% 都没有办法直接应用于类的 "实例" 上面
	% 不过, 具体这个类是什么, 有哪些用处,
	% 以及, 常用类的用法, 我们将在后续的课程中慢慢揭示...
	%
	% 大家现在只要暂时将这些或存入变量,
	% 或直接写下的值(或数据)视为对象,
	% 而这些像绿色高亮部分一样直接对对象进行的操作视作方法就可以了
	%
	% 大家再可以看一下, System.out.println, 中的println,
	% 是不是也是这样的呢, 实际上println也是一个 "方法"
	%
	% 好的, 现在回归主题
	% 大家可以将ppt上的示例实际运行一下试试
	%
	% 我们预期看到的应该是
	% "This is Java, Java is best."
	% "This is Rust, Rust is better."
	% "This is C, they all my kids."
	% (Write it onto the white board)
	%
	% 大家其实从这里就可以看出来,
	% 字符串的拼接实际上不会对原来的字符串有任何更改
	% 而是创建了一个新的字符串
	%
	% 所以, 如果想要修改一个字符串变量的内容, 应该要怎么做?
	% 用赋值符号将其覆盖掉

	\begin{columns}[c]
		\begin{column}{0.9\textwidth}

			\begin{lstlisting}[style=Java]
String str = "String";
str + "foo"; // will produce "Stringfoo"
str`.concat("bar")`; // will produce "Stringbar"
"String".concat("bar"); // will produce "Stringbar"
      \end{lstlisting}


			\begin{lstlisting}[style=Java]
pubic class StringConcate{
  public static void main(String [] args){
    String str = "This is ";
    System.out.println(str.concat("Java, Java is best."));
    System.out.println(str + "Rust, Rust is better.");
    // + will do same as .concat
    System.out.println("This is ".concat("C, they all my kids."));
    System.out.println(str);
  }// ! main
}
      \end{lstlisting}


		\end{column}
	\end{columns}

\end{frame}

\begin{frame}[fragile]

	%% Contents:
	% 
	% 对于一个字符串, 实际上还可以有一些其他的操作
	% 比如, 我们可以通过charAt方法获得字符串中某一个字符
	% 也可以用length方法来获得字符串的长度
	% 不过一定要注意哦, charAt方法的位置是以0开始计数的,
	% 所以如果是charAt(1), 实际上得到的是第二个字符
	% 如下例中, 第二行
	%
	% 字符串也可以被比较,
	% 通过compareTo方法, 可以看出两个字符串是否相同,
	% 如果两个字符串相同, 则会返回一个0值
	% 否则, 会返回一个差值
	% 但是, 其实我们在比较字符串的时候怎么会考虑它们的字典序差异呢?
	% 在要确定是否相等时还要用类似 0 == str.compareTo(str2)的方式
	% 十分的繁琐
	% 所以我们一般会用另一个方法, equals, 来进行比较
	% 这个方法会直接返回两个字符串是否相同, 结果也不是整数值
	% 而是一个逻辑值, 这样就可以被直接用于各种判断了
	%
	% 当然, 这是在判断两个字符串里的字母是否完全一样时用的
	% 那么如果我们希望忽略大小写呢?
	% 那就需要使用compareToIgnoreCase和equalsIgnoreCase两个方法了
	%
	% 当然, 对于空字符串, 其实也有一个判断, 叫做empty
	% 这就避免了写equals("")这样的东西, 也便于理解想要做的事情
	%
	% 当然, 对于字符串的操作, 一定也少不了查找给定内容
	% 这时就需要使用indexOf方法了
	% indexOf会返回第一次找到给定字符串的位置信息,
	% 这个信息和我们提供给charAt的一样, 是个整型数
	%
	% indexOf是从左到右的搜索, 与之相对应的, 还有lastIndexOf
	\begin{columns}
		\begin{column}{0.9\textwidth}

			\begin{lstlisting}[style=Java]
String foo = "Hello, World!";
System.out.println(foo.charAt(1)); // => e
System.out.println(foo.compareTo("hello, world!"));
System.out.println(foo.equalsIgnoreCase("hello, world!"));
System.out.println(foo.indexOf("World"));
      \end{lstlisting}

		\end{column}
	\end{columns}
\end{frame}

\subsubsection{Z --- Byte/Short/Integer/Long}
\begin{frame}[fragile]
	\frametitle{Integer}

	%% Contents:
	%
	% 除了字符串, 计算机既然被称作计算机, 也不可避免会需要与数字打交道,
	% 而整型数就是我们在编程中最长碰见的数字
	%
	% 在上一节课中, 我们已经了解到对于整数的一些基本操作,
	% 比如加减乘除等四则运算
	%
	% 而现在, 我们将更详细的介绍一下我们会遇见的这些整数
	%
	% 我们在上节课中, 见到了int类型
	% 但是, 实际上, 在前面的ppt中, 我们也可以看到
	% 整数一共有4的类型, 分别是byte, short, int, long,
	% 这些不同的类型, 主要差异在存储的值的多少
	% byte最小, 只能存储2^8个数, 也就是说, 从-128到+127的数
	% 其次是short, 2^16,
	% 然后就是我们最常用的int, 2^32, ~2147483648~2147483647
	% 最后是long, 2^64
	%
	% 不过这些东西太过于基础, 只能进行简单的+,-,*,/,%运算
	% 所以, 为了方便, Java将给它们做了一个包装, 使得它们可以像String 
	% 一样用多种方法被操作
	% 这些包装分别为, Byte, Short, Integer, Long,
	%
	% 不过大家要注意, 这些包装, 其实都是类,
	% 类是没有办法通过我们之前见过的各种运算符来操作和比较的
	% 而是需要用一些方法来代行
	% 不过, 对于这些基本类型的包装很简单
	% 实际上它们本身也可以被拆包得到原来的基本类型
	% 所以大家在面对这些东西的时候, 直接转会基本类型再运算就可以啦
	%
	% 而后面我们会详细讲到这些包装类需要如何操作的

	% TODO: With integer operators, like +,-,*,/,%
	\begin{columns}[c]
		\begin{column}{0.9\textwidth}

			\begin{table}
				\begin{tabular}{l l l}
					\toprule
					Type  & Wrapper & Range                                                   \\
					\midrule
					byte  & Byte    & $ -2^{7} ~ 2^{7} - 1 $ (-128 \dots 127)                 \\
					short & Short   & $ -2^{15} ~ 2^{15} - 1 $ (-32768 \dots 32767)           \\
					int   & Integer & $ -2^{31} ~ 2^{31} - 1 $ (-2147483648 \dots 2147483647) \\
					long  & Long    & $ -2^{63} ~ 2^{63} - 1 $ (-9,223,372,036,854,775,808    \\
					      &         & \dots 9,223,372,036,854,775,807)                        \\
					\bottomrule
				\end{tabular}
			\end{table}

			\begin{lstlisting}[style=Java]
int foo = 10;
Integer bar = 10;
String barInString = bar.toString();
System.out.println(bar.equals(foo));
// we should use equals to compare Integers and other wrapper
System.out.println(bar.compareTo(30));
  \end{lstlisting}

		\end{column}
	\end{columns}

\end{frame}

\subsubsection{R --- Float/Double}
\begin{frame}[fragile]
	\frametitle{Real numbers}

	%% Contents:
	%
	% 除了整数, 计算机也可以处理实数
	%
	% 在Java中, 实数一共有两种类型, 分别为float和double
	% 我们称之为, 浮点数和双精度浮点数
	% 浮点数的表示范围远大于整数, 但是浮点数会存在精度问题, float只能处理6~7位有效数字,
	% double则会相对而言多
	%
	% 比如说, 对于float类型, 它最多一共可以精确表示16,777,216个数, 但是它却可以表示1e20这么大的数字
	%
	% 大家可以用如下所示的代码来查看它们的最大最小值
	% 其实, 整数的各个范围也是可以用类似方法来查看最大最小值的哦
	% 具体就是, 包装类 加上 .MIN_VALUE或.MAX_VALUE
	%
	% 与整数类似, 实数也可以进行各种四则运算
	% +,-,*,/
	% 并且, 和整数除法不同的是, 这里的除法并不会放弃余数, 而是会得到小数结果的哦
	% 但是请记住, 取模运算是只可以被用于整数的
	%
	% 当然, 实数也有包装类, 提供了一些方便的用于操作浮点数的方法
	% 还是如代码所示, 分别为Float和Double

	% TODO: +,-,*,/
	\begin{columns}[c]
		\begin{column}{0.9\textwidth}

			\begin{lstlisting}[style=Java]
// float
System.out.println("基本类型:float 二进制位数:" + Float.SIZE);
System.out.println("包装类:java.lang.Float");
System.out.println("最小值:Float.MIN_VALUE=" + Float.MIN_VALUE);
System.out.println("最大值:Float.MAX_VALUE=" + Float.MAX_VALUE);
System.out.println();

// double
System.out.println("基本类型:double 二进制位数:" + Double.SIZE);
System.out.println("包装类:java.lang.Double");
System.out.println("最小值:Double.MIN_VALUE=" + Double.MIN_VALUE);
System.out.println("最大值:Double.MAX_VALUE=" + Double.MAX_VALUE);
System.out.println(); 
      \end{lstlisting}

		\end{column}
	\end{columns}

\end{frame}

\subsubsection{Big numbers --- BigInteger/BitDecimal}
\begin{frame}[fragile]
	\frametitle{Extremely large numbers}

	%% Contents:
	%
	% 虽然看上去long和double可以表示的范围已经足够大了
	% 至少适用于我们日常生活中各种数据的表示了
	%
	% 但是, 在一些特殊的场景之下, 它们还是不够大
	%
	% 这个时候, 就需要所谓的 "大数" 了,
	% Java中常用的大数有两种
	% 第一个是BigInteger, 而另一个是BigDecimal
	% 分别用于表示特大的整数和特大的小数
	% 这两个类型的使用和其他的整数或实数类型及其相似
	% 都是四则运算和一些高级功能

	\begin{columns}[c]
		\begin{column}{0.9\textwidth}

			\begin{lstlisting}[style=Java]
        BigInteger foo = new BigInteger(1);
        BigDecimal bar = new BigDecimal(0.1);
        // Large numbers created through int or float type
        // may cause some unexpedted conditions
        System.out.println("Value of foo is: " + foo);
        System.out.println("Value of bar is: " + bar);
        // We use String to construct them instead
        foo = new BigInteger("9223372036854775809");
        bar = new BigDecimal("0.1");
        System.out.println("Value of bar is: " + bar);
      \end{lstlisting}

		\end{column}
	\end{columns}

\end{frame}

\begin{frame}[fragile]

	%% Contents:
	%
	% 如何创建一个数倒是已经了解了,
	% 但是要如何对它们进行操作呢?
	% 同学们可能会直观想要直接用类似整数类型和实数类型相似的
	% +,-,*,/,或者, 整数类型特有的%, 来对其操作
	% 但是, 实际上这是不可行的
	% 因为这些大数, 实际上也是类,
	% 并且由于其脱离了基本类型, 如int, double,
	% 也就无法用基本类型的操作方式来操作
	% 所以
	% 我们需要用这些类的方法来操作它们
	%
	% 大家可以看到, 这里是一些比较常用的方法
	% add是将一个数与另一个相加
	% compareTo则是相减
	% mutiply相乘
	% devide相除
	% pow乘方
	% 其实还要sqrt开方
	%
	% 当然, 和其他的基本数据的包装类一样,
	% 其实也可以用equals来比较两数是否相等
	%
	% 最后, 在代码中, 我们再次输出了foo的值,
	% 大家其实可以看到
	% foo的值是没有改变的
	% 也就是说, 这些大数和字符串一样, 是 "不变的"
	% 要想修改变量的内容,
	% 就要用结果把原来的东西给覆盖掉
	% UwU~

	\begin{columns}[c]
		\begin{column}{0.9\textwidth}

			\begin{lstlisting}[style=Java]
BigInteger foo = BigInteger("24");
System.out.println(foo.add(BitInteger.ONE));
System.out.println(foo.compareTo(BigInteger.TWO));
System.out.println(foo.mutiply(BigInteger.TWO));
System.out.println(foo.devide(BigInteger.TWO));
System.out.println(foo.mod(BigInteger("9")));
System.out.println(foo.pow(BigInteger.TWO));
System.out.println(foo);
      \end{lstlisting}

		\end{column}
	\end{columns}

\end{frame}

\subsubsection{Logical --- Boolean}
\begin{frame}[fragile]
	\frametitle{Logical representation, Boolean}

	%% Contents:
	%
	% 说了这么久数值计算
	% 也该讲一讲什么是对错了
	% 其实我们生活中很常需要与对错打交道
	% 当然, 写程序也是一样
	% 而且, 计算机有一点好,
	% 就是它总是非黑即白的, 不像我们的生活那样
	% 总是需要考虑什么是不完全正确的, 而什么有需要考量
	%
	% 所以, 在计算机的生活中, 只有是与非
	% true/false
	% 当我们比较两个数字,
	% 如果它们每一位数都相同, 那么它们就一定相等
	% 那么, 一个对于它们是否相等的判断, 就是是
	% 否则, 就只会有否
	%
	% 下面, 就让我们来看一下
	% 计算机中的是与非
	%
	% 在上面的表格中, 我们以经见到了
	% boolean
	% 中文叫做布尔值, 逻辑值
	% 它表示的是一个命题是否为真
	% 比如说, 对于1等于1这个命题, 它得到的就该是true,
	% 大家可以看一下ppt上第一行, 尝试自己运行一下试试看
	% 
	% 这里就要讲到我们的关系运算了
	% 它们用于表示两个基本数值之间的关系
	% 包括了等于判断
	% 大于, 小于, 大于等于, 小于等于 和不等于判断
	%
	% 我们可以把这些判断当做各个命题
	% 如果可以满足这样一个判断, 就是真
	% 否则, 就是假
	%
	% 当然, 我们之前也提到了, 类无法用这些运算来处理
	% 但是, 实际上我们也可以看到,
	% 我们至今遇见的各个类也都提供了比较两个对象的方法
	% 而且, 我们目前为止还是主要和基本数据打交道,
	% 所有不用太担心相关的问题

	% TODO: With relationship operator like ==,<,>,>=,<=,!=, ...
	% With logical operator like ||,&&,!
	\begin{columns}[c]
		\begin{column}{0.9\textwidth}

			\begin{table}
				\begin{tabular}{l l}
					\toprule
					Relationship        & Operator \\
					\midrule
					Equals              & ==       \\
					Greater than        & >        \\
					Less than           & <        \\
					Greater or equal to & >=       \\
					Less to equal to    & <=       \\
					Not Equal           & !=       \\
					\bottomrule
				\end{tabular}
			\end{table}

			\begin{lstlisting}[style=Java]
System.out.println(1==1);
      \end{lstlisting}

		\end{column}
	\end{columns}

\end{frame}
\begin{frame}[fragile]

	%% Contents:
	%
	% 当然咯,
	% 既然其他的基本类型都有可以写出来的类型和变量定义
	% 我们的逻辑值怎么能没有呢
	% 大家可以看到ppt,
	% 这上面分别是它的基本类型定义和包装类定义
	%
	% 大家再想一想, 其他基本类型都有自己的运算规则
	% 逻辑类型是不是也要有自己的运算规则?
	%
	% 当然的,
	% 我们有所谓的逻辑运算来操作逻辑值
	% 比如说, 大家再看ppt
	% 这里就用到了 "与" 运算,
	% 来表示 1 < var < 80这样一个概念

	\begin{columns}[c]
		\begin{column}{0.9\textwidth}

			\begin{lstlisting}[style=Java]
boolean foo = true;
Boolean bar = new Boolean(foo);
      \end{lstlisting}

			\begin{lstlisting}[style=Java]
int var = 20;
System.out.print(1 < var && 80 > var);
      \end{lstlisting}

		\end{column}
	\end{columns}

\end{frame}
\begin{frame}[fragile]

	%% Contents:
	%
	% 大家可以看到下表
	% 这里列出了这些运算符以及它们的运算规则
	% 简单来说, 就是
	% 与运算需要两边都满足, 才得到真
	% 或运算只需要一个满足, 就得到真
	% 否运算, 使得原表达式结果相反

	\begin{columns}[c]
		\begin{column}{0.9\textwidth}

			\begin{table}
				\begin{tabular}{l l l l l l}
					\toprule
					Logic & Operator & Form     & A     & B     & Result \\
					\midrule
					And   & \&\&     & A \&\& B & true  & true  & true   \\
					      &          &          & true  & false & false  \\
					      &          &          & false & true  & false  \\
					      &          &          & false & false & false  \\
					Or    & ||       & A || B   & true  & true  & true   \\                \\
					      &          &          & true  & false & true   \\
					      &          &          & false & true  & true   \\
					      &          &          & false & false & false  \\
					Not   & !        & ! A      & true  &       & false  \\                                            \\
					      &          &          & false &       & true   \\
					\bottomrule
				\end{tabular}
			\end{table}

		\end{column}
	\end{columns}

\end{frame}
\begin{frame}[fragile]

	%% Contents:
	%
	% 有一点大家要注意, 就是, 逻辑运算是 "短路" 的
	% 这是什么意思呢?
	% 就是只要通过前一步操作可以确定整体的值的时候,
	% 就不会再执行后一步了
	% 比如, 当我咋执行一个与运算时,
	% 如果前面以及确定为否了,
	% 那么它就会立即将表达式的整体视为否,
	% 而不执行后一步
	% 相应的, 在执行或运算时,
	% 只要前面确定为真,
	% 则后面就不会再被执行了
	%
	% 大家还记得++运算符是用来做什么的吗?
	% ++运算符是用来让变量自身增加1的,
	% 和各种类的方法不一样,
	% 它的修改会直接反馈到变量本身
	% 所以如果后半句被执行了
	% 我们会期望a的值为6
	% 但是事实上, 我们最终只会得到a的值为5
	%
	% 大家可以试一下这个例子
	% 实际体验一下
	%
	% 不过呢, 我们并不推荐在各种复杂的语句中使用++这样的自增自减运算符
	% 而是直接把它们单列
	% 不过, 这里为了演示逻辑, 所以这样写了
	% 如果对自增运算感兴趣, 可以在网上查询资料
	% 或者在课后单独询问

	\begin{columns}[c]
		\begin{column}{0.9\textwidth}

			\begin{lstlisting}[style=Java]
public class Logical {
  public static void main(String[] args){
    int a = 5;
    boolean b = (a<4) && (a++<10);
    System.out.println("Whole resule is " + b);
    System.out.println("while a is ... :" + a);
  } // ! main
}
      \end{lstlisting}

		\end{column}
	\end{columns}

\end{frame}

\subsection{Bitwise Operation, How those value be calculated in computer}
\begin{frame}[fragile]
	\frametitle{Bitwise Operation}


	% TODO: Byte, Word, Double Word,
	% Structure of IEEE 754 Float
	% sl,sr,xor,or,not(|,&,^,~,<<,>>,>>>)
	\begin{columns}[c]
		\begin{column}{0.9\textwidth}


		\end{column}
	\end{columns}

\end{frame}

%------------------------------------------------

\subsection{Variable Assignment --- Cover origin value, or ``set a value for variable''}
\begin{frame}[fragile]
	\frametitle{Variable Assignment}

	%% Contents
	%
	% 既然讲过了变量的定义, 我们也要重新讲一下变量的赋值
	%
	% 其实在上一节课中, 我们已经简单讲过变量赋值相关的东西了
	% 不知到大家是否还有印象呢
	%

	\begin{columns}[c]
		\begin{column}{0.9\textwidth}


		\end{column}
	\end{columns}

\end{frame}

\subsection{Calculate --- Auto Type Conversion}
\begin{frame}[fragile]
	\frametitle{Type Conversion}

	% Cast from one to another

	% TODO: Integer.parseInt
	% String.toString
	% ...
	\begin{columns}[c]
		\begin{column}{0.9\textwidth}


		\end{column}
	\end{columns}

\end{frame}

\subsection{Calculate --- Explicit Type Conversion}
\begin{frame}[fragile]

	\begin{columns}[c]
		\begin{column}{0.9\textwidth}


		\end{column}
	\end{columns}

\end{frame}

%------------------------------------------------

\section{Input \& Output}
\begin{frame}[fragile]

	\begin{columns}[c]
		\begin{column}{0.9\textwidth}


		\end{column}
	\end{columns}

\end{frame}

\subsection{System.in / System.out}
\begin{frame}[fragile]

	\begin{columns}[c]
		\begin{column}{0.9\textwidth}


		\end{column}
	\end{columns}

\end{frame}

\subsection{System.out.print*}
\begin{frame}[fragile]

	\begin{columns}[c]
		\begin{column}{0.9\textwidth}


		\end{column}
	\end{columns}

\end{frame}

\subsection{java.util.Scanner}
\begin{frame}[fragile]

	\begin{columns}[c]
		\begin{column}{0.9\textwidth}


		\end{column}
	\end{columns}

\end{frame}

%------------------------------------------------

\section{if \dots else \dots --- execute according given condition}
\begin{frame}[fragile]

	\begin{columns}[c]
		\begin{column}{0.9\textwidth}


		\end{column}
	\end{columns}

\end{frame}

\subsection{Logical operation --- when shall it be executed}
\begin{frame}[fragile]

	\begin{columns}[c]
		\begin{column}{0.9\textwidth}


		\end{column}
	\end{columns}

\end{frame}

\subsection{Logical operation --- and what else if the condition is not true}
\begin{frame}[fragile]

	\begin{columns}[c]
		\begin{column}{0.9\textwidth}


		\end{column}
	\end{columns}

\end{frame}


\section{while --- how to execute some sentences once or more}
\begin{frame}[fragile]

	\begin{columns}[c]
		\begin{column}{0.9\textwidth}


		\end{column}
	\end{columns}

\end{frame}

\subsection{Loop --- while condition do something}
\begin{frame}[fragile]

	\begin{columns}[c]
		\begin{column}{0.9\textwidth}


		\end{column}
	\end{columns}

\end{frame}

\subsubsection{Loop --- do something and check wether to continue}
\begin{frame}[fragile]

	\begin{columns}[c]
		\begin{column}{0.9\textwidth}


		\end{column}
	\end{columns}

\end{frame}

\subsubsection{Break --- Jump out of the loop}
\begin{frame}[fragile]

	\begin{columns}[c]
		\begin{column}{0.9\textwidth}


		\end{column}
	\end{columns}

\end{frame}

\subsubsection{Continue --- Next circle}
\begin{frame}[fragile]

	\begin{columns}[c]
		\begin{column}{0.9\textwidth}


		\end{column}
	\end{columns}

\end{frame}






















%------------------------------------------------

\section{Referencing}

\begin{frame}
	\frametitle{Citing References}

	\bigskip % Vertical whitespace

\end{frame}

%------------------------------------------------

\begin{frame} % Use [allowframebreaks] to allow automatic splitting across slides if the content is too long
	\frametitle{References}

	\begin{thebibliography}{99} % Beamer does not support BibTeX so references must be inserted manually as below, you may need to use multiple columns and/or reduce the font size further if you have many references
		\footnotesize % Reduce the font size in the bibliography

		%\bibitem[Kennedy, 2023]{p2}
		%Annabelle Kennedy (2023)
		%\newblock Publication title
		%\newblock \emph{Journal Name} 12(3), 45 -- 678.
	\end{thebibliography}
\end{frame}

%----------------------------------------------------------------------------------------
%  ACKNOWLEDGMENTS SLIDE
%----------------------------------------------------------------------------------------

\begin{frame}
	\frametitle{Acknowledgements}

	\begin{columns}[t] % The "c" option specifies centered vertical alignment while the "t" option is used for top vertical alignment
		\begin{column}{0.45\textwidth} % Left column width
		\end{column}
		\begin{column}{0.5\textwidth} % Right column width
		\end{column}
	\end{columns}
\end{frame}

%----------------------------------------------------------------------------------------
%  CLOSING SLIDE
%----------------------------------------------------------------------------------------

\begin{frame}[plain] % The optional argument 'plain' hides the headline and footline
	\begin{center}
		{\Huge The End}

		\bigskip\bigskip % Vertical whitespace

		{\LARGE Questions? Comments?}
	\end{center}
\end{frame}

%----------------------------------------------------------------------------------------

\end{document}
