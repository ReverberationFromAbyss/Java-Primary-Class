\documentclass[en, 11pt, xcolor=dvipsnames]{beamer}
\usepackage[en]{infocoBeamer}

\definecolor{hrefcol}{RGB}{0, 0, 255} % Example: blue color

% -------------------------------------------------------------------------------------
\usepackage{tikz}
\usetikzlibrary{shapes.callouts, shadows, calc}
\usepackage{listings}

%\usepackage{fontspec-xetex}

\tikzset{note/.style={rectangle collout, rounded corners,fill=grap!20, drop shadow, font=\footnotesize}}

\newcommand{\tikzmark}[1]{\tikz[overlay,remember picture] \node (#1) {};}    

\newcounter{image}
\setcounter{image}{1}

\usepackage{ctex}

\makeatletter
\newenvironment{btHighlight}[1][]
{\begingroup\tikzset{bt@Highlight@par/.style={#1}}\begin{lrbox}{\@tempboxa}}
{\end{lrbox}\bt@HL@box[bt@Highlight@par]{\@tempboxa}\endgroup}

\newcommand\btHL[1][]{%
  \begin{btHighlight}[#1]\bgroup\aftergroup\bt@HL@endenv%
}
\def\bt@HL@endenv{%
  \end{btHighlight}%   
  \egroup
}

\newcommand{\bt@HL@box}[2][]{%
  \tikz[#1]{%
    \pgfpathrectangle{\pgfpoint{0pt}{0pt}}{\pgfpoint{\wd #2}{\ht #2}}%
    \pgfusepath{use as bounding box}%
    \node[anchor=base west,rounded corners, fill=green!30,outer sep=0pt,inner xsep=0.2em, inner ysep=0.1em,  #1](a\theimage){\usebox{#2}};
  }%
   %\tikzmark{a\theimage} <= can be used, but it leads to a spacing problem
   % the best approach is to name the previous node with (a\theimage)
 \stepcounter{image}
}
\makeatother

%\setmainfont{FiraCode Nerd Font Mono}
%\setsansfont{FiraCode Nerd Font Mono}
%\setmonofont{FiraCode Nerd Font Mono}

%\setCJKmainfont[BoldFont = STLibian-SC-Regular,]{TpldKhangXiDictTrial}
%\setCJKsansfont{DFWaWaSC-W5}
%\setCJKmonofont{STXingkai-SC-Light}
%\setCJKfamilyfont{qingsong}{FZQKBYSJW--GB1-0}
%\lstset{language=Java,
%        basicstyle=\footnotesize\ttfamily,
%        keywordstyle=\footnotesize\color{blue}\ttfamily,
%        moredelim=**[is][\bthl]{`}{`},
%}

\lstset{
  basicstyle            =   \scriptsize\ttfamily,          % 基本代码风格
    keywordstyle        =   \bfseries,          % 关键字风格
    commentstyle        =   \scriptsize\tt\itshape,  % 注释的风格,斜体
    stringstyle         =   \tt\itshape,  % 字符串风格
    flexiblecolumns,                % 别问为什么,加上这个
    numbers             =   left,   % 行号的位置在左边
    showspaces          =   false,  % 是否显示空格,显示了有点乱,所以不现实了
    numberstyle         =   \ttfamily,    % 行号的样式,小五号,tt等宽字体
    showstringspaces    =   false,
    captionpos          =   t,      % 这段代码的名字所呈现的位置,t指的是top上面
    frame               =   lrtb,   % 显示边框
}

\lstdefinestyle{Java}{
    language        =   Java,
    commentstyle    =   \scriptsize\color{red}\ttfamily,
    basicstyle      =   \scriptsize\ttfamily,
    keywordstyle    =   \scriptsize\color{blue}\bfseries,
    keywordstyle    =   [2] \color{teal},
    stringstyle     =   \scriptsize\color{magenta}\tt,
    numberstyle     =   \tiny\ttfamily,
    showstringspaces=   false,
    flexiblecolumns,                % 别问为什么,加上这个
    breaklines      =   true,   % 自动换行,建议不要写太长的行
    columns         =   fixed,  % 如果不加这一句,字间距就不固定,很丑,必须加
    basewidth       =   0.15cm,
		moredelim       =   **[is][\btHL]{`}{`},
    captionpos      =   t,      % 这段代码的名字所呈现的位置,t指的是top上面
    frame           =   lrtb,   % 显示边框
}
% -------------------------------------------------------------

% meta-data
\title{Java Programming Language \\ Simple Guide} % The short title in the optional parameter appears at the bottom of every slide, the full title in the main parameter is only on the title page
\subtitle{Infoco Programming Classes} % Presentation subtitle, remove this command if a subtitle isn't required
\author{Ug. Sihang Sha \and Pg. Muxi Qiao} % Presenter name(s), the optional parameter can contain a shortened version to appear on the bottom of every slide, while the main parameter will appear on the title slide
\institute{Xiann' Jiaotong Livepool University \\ \smallskip \textit{infoco@xjtlu.edu.cn}} % Your institution, the optional parameter can be used for the institution shorthand and will appear on the bottom of every slide after author names, while the required parameter is used on the title slide and can include your email address or additional information on separate lines
\date{\today} % Presentation date or conference/meeting name, the optional parameter can contain a shortened version to appear on the bottom of every slide, while the required parameter value is output to the title slide

%-------------------------------------------------------------------------------------------------------

% document body
% ------------------------------------------------------------------------------------------------------
\begin{document}

\maketitle

% ------------------------------------------------------------------------------------------------------

%------------------------------------------------
\section{Sort}
\begin{frame}[fragile]
	\frametitle{Bubble Sort}


	\begin{columns}[c]
		\begin{column}{0.98\textwidth}

			在上一次的课程中, 我们了解到了线性数据结构

			以及它们的操作方式

			这节课, 我们将来实际了解一下, 排序

			排序是对这些线性数据结构中无序的数据进行排列, 是其变得有序的操作
			排序的用处很多, 比如, 我们需要按照平均成绩来获取哪些学生学习努力,
			可以获得奖学金

		\end{column}
	\end{columns}

\end{frame}

\subsection{Bubble Sort}
\begin{frame}[fragile]
	\frametitle{Bubble Sort}


	\begin{columns}[c]
		\begin{column}{0.98\textwidth}
			冒泡排序是最简单的排序方式之一

			但是为此它的效率也并不高

		\end{column}
	\end{columns}

\end{frame}

\begin{frame}[fragile]


	\begin{columns}[c]
		\begin{column}{0.98\textwidth}

			\begin{itemize}
				\item 比较相邻的元素。如果第一个比第二个大,就交换他们两个。
				\item 对每一对相邻元素作同样的工作,从开始第一对到结尾的最后一对。这步做完后,最后的元素会是最大的数。
				\item 针对所有的元素重复以上的步骤,除了最后一个。
				\item 持续每次重复上面的步骤,直到没有任何一对数字需要比较。
			\end{itemize}


		\end{column}
	\end{columns}

\end{frame}

\subsection{Quick Sort}
\begin{frame}[fragile]
	\frametitle{Quick Sort}

	%% Contents:


	\begin{columns}[c]
		\begin{column}{0.98\textwidth}

			因为冒泡排序的效率实在太过于惨不忍睹, 所以就有人研究出一个方法
			可以极大的提高对于数据的排序效率

			这种排序算法就被称作, 快速排序

			顾名思义, 这个算法的排序速度很快

		\end{column}
	\end{columns}

\end{frame}

\begin{frame}[fragile]
	%% Contents:


	\begin{columns}[c]
		\begin{column}{0.98\textwidth}

			\begin{itemize}
				\item 从数列中挑出一个元素,称为 "基准"(pivot);
				\item 重新排序数列,所有元素比基准值小的摆放在基准前面,所有元素比基准值大的摆在基准的后面(相同的数可以到任一边)。在这个分区退出之后,该基准就处于数列的中间位置。这个称为分区(partition)操作;
				\item 然后再对两个分区进行如上操作
			\end{itemize}

		\end{column}
	\end{columns}

\end{frame}

\subsection{Bucket Sort}
\begin{frame}[fragile]
	\frametitle{Bucket Sort}

	%% Contents:

	\begin{columns}[c]
		\begin{column}{0.98\textwidth}

			假设一个学校中有2000个同学, 需要将他们的身高排序
			请实现一个算法:

			保证, 只有整数, 以厘米为单位, 范围在150到200

		\end{column}
	\end{columns}
\end{frame}

\begin{frame}[fragile]

	%% Contents:

	\begin{columns}[c]
		\begin{column}{0.98\textwidth}

			我们可知, 与上方两个排序所处理的情况不同

			在该例中, 我们面对的是数据规模极大, 数据重复度极高的数据

			如果采用上述两个算法, 几乎都可以被认为是极差的数据输入
			会使得总体算法复杂度趋于最差情况

			所以, 我们可以转变思路:
			既然数据的重复度很高, 那么为什么不先统计这些数据都出现了几次,
			再统一输出呢?

			这就是桶排序

		\end{column}
	\end{columns}
\end{frame}

\begin{frame}[fragile]

	%% Contents:

	\begin{columns}[c]
		\begin{column}{0.98\textwidth}

			\begin{itemize}
				\item 扫描整个输入, 分别对出现的数据计数
				\item 然后按照单个数据的大小排序
				\item 依照计数器数组记录的数据个数, 生成结果数组
			\end{itemize}

		\end{column}
	\end{columns}
\end{frame}
% -----------------------------------------------

\section{Read A Number}
\subsection{Integer only}
\begin{frame}[fragile]
	\frametitle{Parse A Integer}

	%% Contents

	\begin{columns}[c]
		\begin{column}{0.98\textwidth}

			检查第一位是否为正号或者负号, 如果为正号, 保持符号位不变
			如果为负号, 将符号记为负

			开始依次读取后方的数字, 更新缓存的数值

			如果读到非数字或输入已经抵达结尾, 意味着数字已经结束

			返回

		\end{column}
	\end{columns}
\end{frame}

\subsection{Real Numbers}
\begin{frame}[fragile]
	\frametitle{How About Reals?}

	%% Contents:


	\begin{columns}[c]
		\begin{column}{0.98\textwidth}

			与上面类似,
			我们需要检查第一位是否为正号或负号

			读取, 若为数字, 更新缓存的数值
			若为小数点, 记录小数点位置信息

			读到非数字或者输出抵达结尾, 返回

		\end{column}
	\end{columns}

\end{frame}

% -------------------------------------------------

\section{Game Of Life}
\begin{frame}[fragile]
	\frametitle{Game Of Life --- Your First Project}

	%% Contents:

	\begin{columns}[c]
		\begin{column}{0.98\textwidth}

			康威生命游戏:

			有如下规则:

			游戏进行在一个棋盘中, 每一个棋盘格对应一个细胞, 细胞只有两种状态
			生, 或者, 死, 对于每一代, 我们根据一定规则, 迭代整个棋盘

			对于整个棋盘:

			\begin{itemize}
				\item 长宽相等
				\item 记录了当前每个细胞的状态
				\item 有一个初始状态
			\end{itemize}

			对于每一个细胞:
			\begin{itemize}
				\item (1)当前细胞为死亡状态时,当周围有3个存活细胞时,则迭代后该细胞变成存活状态(模拟繁殖);若原先为生,则保持不变。
				\item (2)当前细胞为存活状态时,当周围的邻居细胞低于两个(不包含两个)存活时,该细胞变成死亡状态(模拟生命数量稀少)。
				\item (3)当前细胞为存活状态时,当周围有两个或3个存活细胞时,该细胞保持原样。
				\item (4)当前细胞为存活状态时,当周围有3个以上的存活细胞时,该细胞变成死亡状态(模拟生命数量过多)。
			\end{itemize}


		\end{column}
	\end{columns}
\end{frame}

\begin{frame}[fragile]

	%% Contents:

	\begin{columns}[c]
		\begin{column}{0.98\textwidth}

			定义一个棋盘, 用数组存储:

			将元素视为细胞, 有两个状态

			更新规则如何?

		\end{column}
	\end{columns}
\end{frame}

\QApage

% ----------------------------------------------------------------------

\section{Referencing}

\begin{frame}
	\frametitle{Citing References}

	\bigskip % Vertical whitespace

\end{frame}

% --------------------


% ----------------------------------------------------------------------

\begin{frame}
	\frametitle{Acknowledgements}

	\begin{columns}[t] % The "c" option specifies centered vertical alignment while the "t" option is used for top vertical alignment
		\begin{column}{0.45\textwidth} % Left column width
		\end{column}
		\begin{column}{0.5\textwidth} % Right column width
		\end{column}
	\end{columns}
\end{frame}

\end{document}
